\documentclass[10pt]{article}
\usepackage[utf8]{inputenc}
\usepackage[T1]{fontenc}
\usepackage{amsmath}
\usepackage{amsfonts}
\usepackage{amssymb}
\usepackage[version=4]{mhchem}
\usepackage{stmaryrd}

\title{Calendar Year }

\author{}
\date{}


\begin{document}
\maketitle
Let's begin by exploring the data preparation portion of ratemaking. As we saw in the fundamental insurance equation, one main component of ratemaking is losses. By understanding losses, pricing actuaries (i.e. actuaries who set rates) can more accurately estimate the rates that need to be charged for an insurance policy. For ratemaking purposes, these losses need to be aggregated, developed, and trended.

In Section S5.1, we aggregated losses by accident years. There are other ways to aggregate loss data, which we will discuss here.

Loss data can be aggregated using four common methods: calendar year, accident year, policy year, and report year. Report year aggregation (mainly used for claimsmade policies) is not covered on this exam.

\begin{itemize}
  \item Calendar year aggregation considers all loss transactions that occur in a twelve-month calendar year from January 1 to December 31.

  \item Accident year aggregation considers all loss transactions for accidents that occur within a twelve-month period regardless of the policy issuance date or the claim report date.

  \item Policy year aggregation considers all loss transactions for policies that are issued within a twelve-month period regardless of the accident occurrence date or the claim report date.

\end{itemize}

For accident year aggregation and policy year aggregation, the twelve-month period does not need to be a calendar year. It can be any period that is twelve months long, e.g. May 1 to April 30. Unless stated otherwise, the default is to assume January 1 to December 31 during the exam.

Now, we will go into more detail on how losses are aggregated under each method. For this, let $L_{i}^{P}$ and $L_{i}^{I}$ represent the paid losses and incurred losses, respectively, for time period $i$.

For calendar year aggregation, the calendar year paid losses equal the sum of losses paid throughout the year. Then, the calendar year incurred losses equal the sum of the calendar year paid losses and the change in reserves during the year. So, for calendar year $i$, this can be expressed as:

$$
L_{i}^{I}=L_{i}^{P}+R_{i}-R_{i-1}
$$

where $R_{i}$ is the reserves at the end of calendar year $i$. Note that these losses will be final at the end of a calendar year because any further development will be recorded in the following calendar year.

\section{Accident Year}
For a particular accident year, losses are not fixed and often change as claims are paid and reserves are adjusted. Therefore, for accident year aggregation, a valuation date is required. The valuation date plays the role of the development year introduced in the chain-ladder method.

Then, the accident year paid losses equal the sum of losses paid as of the valuation date on accidents that occurred in that accident year.

The accident year incurred losses equal the sum of accident year paid losses and the reserves as of the valuation date. Thus:

$$
L_{i}^{I}=L_{i}^{P}+R_{i}
$$

where $R_{i}$ is the reserves as of the valuation date for accidents that happened in accident year $i$.

\section{Policy Year}
Policy year paid losses and incurred losses are calculated like accident year paid losses and incurred losses. The only difference is that losses are aggregated based on the twelve-month period during which the policies were issued, rather than the period in which the accidents occurred. Let's apply these aggregation methods to an example to get a better idea of how they work.

You are given:

\begin{itemize}
  \item Sigma Insurance Company issued an auto policy on July 1, 2015.

  \item A claim on this policy was reported for an accident that occurred on February 2, 2016.

  \item Below are the transactions for this claim:

\end{itemize}

\begin{center}
\begin{tabular}{|c|c|c|}
\hline
Transaction Date & Incremental Paid Loss & Case Reserve \\
\hline
$2 / 15 / 2016$ & 0 & 1,000 \\
\hline
$11 / 15 / 2016$ & 0 & 2,600 \\
\hline
$3 / 1 / 2017$ & 500 & 2,100 \\
\hline
$8 / 1 / 2017$ & 1,000 & 1,100 \\
\hline
$12 / 1 / 2017$ & 0 & 3,000 \\
\hline
\end{tabular}
\end{center}

\begin{itemize}
  \item Sigma Insurance Company only writes annual insurance policies.
\end{itemize}

Calculate the incurred loss for this claim for:

\begin{enumerate}
  \item CY2016 and CY2017.

  \item AY2016 as of $12 / 31 / 2016$ and AY2016 as of $12 / 31 / 2017$.

  \item PY2015 as of 12/31/2017 and PY2016 as of 12/31/2017.

\end{enumerate}

We will assume that this is the only claim reported to Sigma Insurance Company and that the reserve on January 1, 2015 is zero.

\section{Calendar Year}
To calculate the incurred loss for CY2016, consider the transactions in 2016. Nothing is paid in 2016, but there are two case reserves. However, to calculate the change in reserves, we only need the case reserves at the end of 2015 and 2016. Thus, any reserve adjustments at any other time during the year are irrelevant. At the end of 2016, the case reserve is 2,600. At the end of 2015, the case reserve is zero. Therefore, the incurred loss for CY2016 is:

$$
\begin{aligned}
L_{\mathrm{CY} 2016}^{I} & =L_{\mathrm{CY} 2016}^{P}+R_{\mathrm{CY} 2016}-R_{\mathrm{CY} 2015} \\
& =0+2,600-0 \\
& =\mathbf{2 , 6 0 0}
\end{aligned}
$$

In 2017, the company pays a total of 1,500 . At the end of 2017 , the case reserve is 3,000 . Thus, the incurred loss for CY2017 is:

$$
\begin{aligned}
L_{\mathrm{CY} 2017}^{I} & =L_{\mathrm{CY} 2017}^{P}+R_{\mathrm{CY} 2017}-R_{\mathrm{CY} 2016} \\
& =1,500+3,000-2,600 \\
& =\mathbf{1 , 9 0 0}
\end{aligned}
$$

\section{Accident Year}
The accident occurs in 2016. So, all loss transactions related to this claim will contribute to AY2016. Nothing is paid in 2016, but the case reserve as of $12 / 31 / 2016$ is 2,600 . Thus, as of $12 / 31 / 2016$ :

$$
\begin{aligned}
L_{\mathrm{AY} 2016}^{I} & =L_{\mathrm{AY} 2016}^{P}+R_{\mathrm{AY} 2016} \\
& =0+2,600 \\
& =\mathbf{2 , 6 0 0}
\end{aligned}
$$

As of $12 / 31 / 2017$, the company has paid a total of 1,500 , and the case reserve is 3,000 . Therefore, the incurred loss for AY2016 as of 12/31/2017 is:

$$
\begin{aligned}
L_{\mathrm{AY} 2016}^{I} & =L_{\mathrm{AY} 2016}^{P}+R_{\mathrm{AY} 2016} \\
& =1,500+3,000 \\
& =\mathbf{4 , 5 0 0}
\end{aligned}
$$

\section{Policy Year}
Since the policy is written in 2015 , all loss transactions for this policy will contribute to PY2015. As of 12/31/2017, the company has paid a total of 1,500 , and the case reserve is 3,000 . Therefore, the incurred loss for PY2015 as of $12 / 31 / 2017$ is:

$$
\begin{aligned}
L_{\mathrm{PY} 2015}^{I} & =L_{\mathrm{PY} 2015}^{P}+R_{\mathrm{PY} 2015} \\
& =1,500+3,000 \\
& =\mathbf{4 , 5 0 0}
\end{aligned}
$$

Because the policy is written in 2015 , none of the transactions will contribute to PY2016. Therefore, the incurred loss for PY2016 as of 12/31/2017 is:

$$
L_{\mathrm{PY} 2016}^{I}=\mathbf{0}
$$

\section{Example S5.2.1.1}
You are given the following information on Universal Insurance Company's policies and claims:

\begin{center}
\begin{tabular}{|c|c|c|}
\hline
Policy & Effective Date & Expiration Date \\
\hline
A & $1 / 1 / 2017$ & $12 / 31 / 2017$ \\
\hline
B & $4 / 1 / 2017$ & $3 / 31 / 2018$ \\
\hline
C & $2 / 1 / 2018$ & $1 / 31 / 2019$ \\
\hline
\end{tabular}
\end{center}

\begin{center}
\begin{tabular}{|c|c|c|}
\hline
Claim & Policy & Accident Date \\
\hline
$\# 1$ & B & $6 / 28 / 2017$ \\
\hline
$\# 2$ & A & $12 / 20 / 2017$ \\
\hline
$\# 3$ & C & $7 / 14 / 2018$ \\
\hline
\end{tabular}
\end{center}

\begin{center}
\begin{tabular}{|c|c|c|c|}
\hline
Claim & Transaction Date & Incremental Paid Loss & Case Reserve \\
\hline
$\# 1$ & $7 / 2 / 2017$ & 0 & 4,000 \\
\hline
$\# 1$ & $7 / 20 / 2017$ & 2,300 & 1,700 \\
\hline
$\# 1$ & $8 / 15 / 2017$ & 1,700 & 0 \\
\hline
$\# 2$ & $12 / 28 / 2017$ & 0 & 3,000 \\
\hline
$\# 2$ & $1 / 3 / 2018$ & 0 & 5,000 \\
\hline
$\# 2$ & $6 / 6 / 2018$ & 3,200 & 1,800 \\
\hline
\end{tabular}
\end{center}

\begin{center}
\begin{tabular}{|c|c|c|c|}
\hline
Claim & Transaction Date & Incremental Paid Loss & Case Reserve \\
\hline
$\# 3$ & $7 / 30 / 2018$ & 0 & 2,500 \\
\hline
$\# 3$ & $8 / 15 / 2018$ & 1,400 & 1,100 \\
\hline
$\# 2$ & $9 / 10 / 2018$ & 1,500 & 1,000 \\
\hline
\end{tabular}
\end{center}

The loss reserve on 12/31/2016 is zero, and Universal does not have any other claims.

Calculate the incurred losses for all claims for:

\begin{enumerate}
  \item calendar year 2018.

  \item accident year 2017 as of $12 / 31 / 2017$.

  \item policy year 2018 as of $12 / 31 / 2018$.

\end{enumerate}

\section{Solution to (1)}
To calculate incurred losses for CY2018, we need the paid losses for CY2018 and the change in reserves during the year. The paid losses in 2018 are:

$$
\begin{aligned}
L_{\mathrm{CY} 2018}^{P} & =3,200+1,400+1,500 \\
& =6,100
\end{aligned}
$$

Then, calculate the reserves at the end of CY2017 for each claim. At the end of CY2017, claims \#1 and \#3 do not have case reserves. Claim \#2 has a case reserve of 3,000 . Thus:

$$
R_{\mathrm{CY} 2017}=3,000
$$

Next, calculate the reserves at the end of CY2018 for each claim. At the end of CY2018, claims \#1 does not have a case reserve. Claim \#2 has a case reserve of 1,000 . Claim \#3 has a case reserve of 1,100 . Thus:

$$
R_{\mathrm{CY} 2018}=2,100
$$

Therefore, the incurred losses for CY2018 are:

$$
\begin{aligned}
L_{\mathrm{CY} 2018}^{I} & =L_{\mathrm{CY} 2018}^{P}+R_{\mathrm{CY} 2018}-R_{\mathrm{CY} 2017} \\
& =6,100+2,100-3,000 \\
& =\mathbf{5 , 2 0 0}
\end{aligned}
$$

\section{Solution to (2)}
To calculate the incurred losses for AY2017, we only consider loss transactions from accidents that occurred in 2017. Focus on the second table in the question. Only claims \#1 and \#2 are from accidents occurring in 2017. We can ignore claim \#3.

As of $12 / 31 / 2017$, claim \#1 is fully paid, and no payments have been made on claim \#2.

$$
\begin{aligned}
L_{\mathrm{AY} 2017}^{P} & =2,300+1,700 \\
& =4,000
\end{aligned}
$$

In addition, claim \#1 has no case reserve because it is fully paid, and claim \#2 has a case reserve of 3,000 .

$$
R_{\mathrm{AY} 2017}=3,000
$$

Therefore, the incurred losses for AY2017 as of 12/31/2017 are:

$$
\begin{aligned}
L_{\mathrm{AY} 2017}^{I} & =L_{\mathrm{AY} 2017}^{P}+R_{\mathrm{AY} 2017} \\
& =4,000+3,000 \\
& =\mathbf{7 , 0 0 0}
\end{aligned}
$$

\section{Solution to (3)}
To calculate the losses for PY2018, we only consider loss transactions for policies issued in 2018. Based on the first table in the question, only Policy C is written in 2018. Claim \#3 is associated with this policy.

As of the end of 2018, the company has paid 1,400 and has a case reserve of 1,100 for claim \#3. Thus, as of 12/31/2018:

$$
\begin{aligned}
L_{\mathrm{PY} 2018}^{I} & =L_{\mathrm{PY} 2018}^{P}+R_{\mathrm{PY} 2018} \\
& =1,400+1,100 \\
& =\mathbf{2}, \mathbf{5 0 0}
\end{aligned}
$$

For ratemaking, accident year and policy year are the most common methods of aggregating losses. Accident year is often used because the data then reflects a more responsive loss experience, as the data tends to be more recent. Policy year data is typically used because it reflects a more complete experience. However, it may be less responsive, as the data will likely be outdated. On the other hand, calendar year data is generally not used in ratemaking because there is often a mismatch in timing between the premiums and losses.


\end{document}